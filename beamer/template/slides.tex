%%%%%%%%%%%%%%%%%%%%%%%%%%%%%%%%%%%%%%%%%
% Udesc presentation using Execushares
% LaTeX Template
% Version 1.0 (2017-12-22)
%
% Authors: Jackson Antonio do Prado Lima 
% 		  <jacksonpradolima at gmail dot com>
%
%%%%%%%%%%%%%%%%%%%%%%%%%%%%%%%%%%%%%%%%%

\documentclass[aspectratio=169]{beamer}
\usepackage[utf8]{inputenc}
\usepackage[brazil]{babel}

\usetheme[colortheme=green]{Udesc}

\usepackage{blindtext}
\usepackage{moresize}
\usepackage{xcolor}
\usepackage{colortbl}
\usepackage{booktabs}

\usepackage[ruled,onelanguage]{algorithm2e}
\usepackage[backend=bibtex,style=numeric]{biblatex}
\bibliography{refs}

\title{Consultas simples para recuperação de informações em SQL}
\author{Jackson Antonio do Prado Lima (jacksonpradolima at gmail dot com)}
\date{13 de dezembro de 2017}

\subtitle{\tiny{CEPLAN, São Bento do Sul, Brasil}}

\setcounter{showSlideNumbers}{1}

\begin{document}
	\frame{\titlepage}

	%Agenda
	\begin{frame}		
		\frametitle{Contents}
		\begin{enumerate}
			\item O que foi visto na última aula?  \\
			\item Consultas simples em SQL \\
			\item O que teremos na próxima aula? \\
			\item Exercícios  \\
		\end{enumerate}
	\end{frame}

	\startprogressbar
	
	%Sections
	\section{O que foi visto na última aula?}
		\begin{frame}
			\frametitle{Na última aula}
			\begin{enumerate}
				\item O que é SQL (\textit{Structured Query Language})
				\begin{itemize}
					\item Uma linguagem de BD
					\item Instruções para definições de dados, consultas e atualizações
				\end{itemize}
				\item Comandos de DDL (\textit{Data Definition Language}) da SQL para criação, modificação, remoção de tabelas e seus atributos, além dos tipos de dados básicos em SQL
				\item Comandos de DML (\textit{Data Manipulation Language}) da SQL para inserção, atualização e deleção de tuplas das tabelas 
			\end{enumerate}
		\end{frame}

		\section{Consultas simples em SQL}
			\begin{frame}{Contexto}
				\begin{itemize}
					\item Alguns autores costumam denominar como DQL (\textit{Data Query Language})
					\item Instrução básica de recuperação de informações do BD: \textbf{SELECT}
				\end{itemize}
			\end{frame}
		
			\begin{frame}[fragile]{SELECT-FROM-WHERE}
				\begin{block}{Forma básica}
							{\ttfamily
							SELECT <lista atributos>\\
							FROM <lista tabelas>\\
							WHERE <condição>
							}
				\end{block}
				\noindent onde:
				\begin{itemize}
					\item \textbf{lista atributos} é uma lista de nomes de atributo cujos valores devem ser recuperados pela consulta.
					\item \textbf{lista tabelas} é uma lista dos nomes de relação exigidos para processar a consulta.
					\item \textbf{condição} é uma expressão condicional (booleana) que identifica as tuplas a serem recuperadas pela consulta.
				\end{itemize}
			\end{frame}

			\begin{frame}[fragile]{SELECT-FROM-WHERE}
				\begin{block}{\textbf{PRODUTO}}
					\begin{table}
						\begin{tabular}{|c|l|c|c|c|}
							\toprule
							\textbf{id} & \textbf{pnome} & \textbf{preco} & \textbf{categoria} & \textbf{quantidade} \\ \midrule
							     1      & lapis          &     19.99      &     papelaria      &         30          \\ \midrule
							     2      & lapiseira      &     29.99      &     papelaria      &         55          \\ \midrule
							     3      & camera         &     103.99     &     fotografia     &         10          \\ \midrule
							     4      & televisao      &     203.99     &    eletronicos     &         15          \\ \bottomrule
						\end{tabular}
					\end{table}
				\end{block}
			\end{frame}
		
			\begin{frame}[fragile]{SELECT-FROM-WHERE}
				\begin{block}{Exemplo 1}
					{\ttfamily
						SELECT *\\
						FROM Produto
					}
				\end{block}
				\begin{table}
					\begin{tabular}{|c|l|c|c|c|}
						\toprule
						\textbf{id} & \textbf{pnome} & \textbf{preco} & \textbf{categoria} & \textbf{quantidade} \\ \midrule
						1      & lapis          &     19.99      &     papelaria      &         30          \\ \midrule
						2      & lapiseira      &     29.99      &     papelaria      &         55          \\ \midrule
						3      & camera         &     103.99     &     fotografia     &         10          \\ \midrule
						4      & televisao      &     203.99     &    eletronicos     &         15          \\ \bottomrule
					\end{tabular}
				\end{table}
			\end{frame}
			
			\begin{frame}[fragile]{SELECT-FROM-WHERE}
				\begin{block}{Exemplo 2}
					{\ttfamily
						SELECT *\\
						FROM Produto\\
						WHERE categoria = ``papelaria''
					}
				\end{block}
				\begin{table}
					\begin{tabular}{|c|l|c|c|c|}
						\toprule
						\textbf{id} & \textbf{pnome} & \textbf{preco} & \textbf{categoria} & \textbf{quantidade} \\ \midrule
						     1      & lapis          &     19.99      &     papelaria      &         30          \\ \midrule
						     2      & lapiseira      &     29.99      &     papelaria      &         55          \\ \bottomrule
					\end{tabular}
				\end{table}
			\end{frame}
		
			\begin{frame}[fragile]{SELECT-FROM-WHERE}
				\begin{block}{Exemplo 3}
					{\ttfamily
						SELECT *\\
						FROM Produto\\
						WHERE preco > 100
					}
				\end{block}
				\begin{table}
						\begin{tabular}{|c|l|c|c|c|}
							\toprule
							\textbf{id} & \textbf{pnome} & \textbf{preco} & \textbf{categoria} & \textbf{quantidade} \\ \midrule
							     3      & camera         &     103.99     &     fotografia     &         10          \\ \midrule
							     4      & televisao      &     203.99     &    eletronicos     &         15          \\ \bottomrule
						\end{tabular}
				\end{table}
			\end{frame}
	
			\begin{frame}{Eliminação de Duplicações}
				Devido a não haver eliminação de duplicatas no \textbf{SELECT} utilizamos o \textbf{DISTINCT}.
				\BlankLine
				\begin{columns}[c]
					\begin{column}[c]{.48\textwidth}	
						{\ttfamily
							SELECT DISTINCT categoria\\
							FROM Produto\\
						}
						
						\begin{table}
							\begin{tabular}{|c|}
								\toprule
								\textbf{categoria} \\ \midrule
								    papelaria      \\ \midrule
								    fotografia     \\ \midrule
								   eletronicos     \\ \bottomrule
							\end{tabular}
						\end{table}
					\end{column}
			
					\begin{column}[c]{.48\textwidth}	\\
						{\ttfamily
							SELECT categoria\\
							FROM Produto\\
						}
						
						\begin{table}
							\begin{tabular}{|c|}
								\toprule
								\textbf{categoria} \\ \midrule
								    papelaria      \\ \midrule
								    papelaria      \\ \midrule
								    fotografia     \\ \midrule
								   eletronicos     \\ \bottomrule
							\end{tabular}
						\end{table}
					\end{column}
				\end{columns}
			\end{frame}
	
	
			\begin{frame}{Ordenação do Resultado}
				\begin{itemize}
					\item A ordenação é ascendente a não ser que a palavra \textbf{DESC} seja especificada.
					\item No caso de igualdade no primeiro atributo da cláusula \textbf{ORDER BY}, o segundo atributo é utilizado e assim por diante.
				\end{itemize}
			
				{\ttfamily
					SELECT pnome, preco, quantidade\\
					FROM Produto\\
					WHERE categoria=``papelaria'' AND preco > 50\\
					ORDER BY preco DESC, pnome ASC
				}
			\end{frame}
	
			\begin{frame}{Exercício de fixação}
				\begin{enumerate}
					\item O resultado ao aplicar a consulta: {\ttfamily SELECT categoria FROM Produto ORDER BY categoria}
					\item O resultado ao aplicar a consulta: {\ttfamily SELECT DISTINCT categoria FROM Produto ORDER BY categoria DESC}
				\end{enumerate}
			
%				\BlankLine\BlankLine
%				\textcolor{red}{\texttt{*Entregar no final da aula =D}}
			\end{frame}
	
			\begin{frame}{Facilidades na projeção dos resultados}
				\BlankLine\BlankLine
				\begin{itemize}
					\item Uso de aritmética nas consultas (+, - , *, /)
					\item Renomeação
				\end{itemize}
				
				\begin{block}{Exemplo}
					{\ttfamily
						SELECT pnome AS Nome, (preco*quantidade) AS Total\\
						FROM Produto\\
					}
				\end{block}
				\begin{table}
					\begin{tabular}{|l|c|}
						\toprule
						\textbf{Nome} & \textbf{Total} \\ \midrule
						lapis         &     599.7      \\ \midrule
						lapiseira     &    1649.45     \\ \midrule
						camera        &     1039.9     \\ \midrule
						televisao     &    3059.85     \\ \bottomrule
					\end{tabular}
				\end{table}
			\end{frame}
			
			\begin{frame}{Funções de agregação}
				\BlankLine\BlankLine
				SQL dá suporte a diversas operações de agregação:
				\begin{itemize}
					\item \textbf{SUM}: soma dos valores de um atributo
					\item \textbf{MIN}: valor mínimo de um atributo
					\item \textbf{MAX}: valor máximo de um atributo
					\item \textbf{AVG}: média dos valores de um atributo
					\item \textbf{COUNT}: contagem de ocorrências [de um atributo]
				\end{itemize}

				Observações:
				\begin{itemize}
					\item Com exceção de \textbf{COUNT}, todas as outras operações de agregação aplicam-se a um único atributo.
					\item \textbf{COUNT} não conta valores nulos.
				\end{itemize}			
			\end{frame}
		
			\begin{frame}{Funções de agregação}
				\begin{block}{Exemplo}
					{\ttfamily
						SELECT COUNT(pnome) AS count\_nome, MIN(preco) AS min\_preco, MAX(preco) AS max\_preco, AVG(quantidade) AS avg\_qtd, SUM(quantidade), sum\_qtd\\
						FROM Produto\\
					}
				\end{block}
				\begin{table}
					\begin{tabular}{|c|c|c|c|c|}
						\toprule
						\textbf{count\_nome} & \textbf{min\_preco} & \textbf{max\_preco} & \textbf{avg\_qtd} & \textbf{sum\_qtd} \\ \midrule
						         4           &        19.99        &       203.99        &       27.5        &        110        \\ \bottomrule
					\end{tabular}
				\end{table}
			\end{frame}
		
			\begin{frame}{Funções de agregação}
				\begin{block}{Exemplo com \textbf{DISTINCT}}
					{\ttfamily
						SELECT COUNT(DISTINCT pnome) AS count\_nome\\
						FROM Produto\\
					}
				\end{block}
			\end{frame}
		
			\begin{frame}{Incrementando a cláusula WHERE}
				\begin{itemize}
					\item Busca por padrões: \textbf{\textit{Cláusula} [NOT] LIKE}
					\item Teste de existência de valores nulos: \textbf{\textit{Cláusula} IS [NOT] NULL}
					\item Busca por intervalos de valores: \textbf{\textit{Cláusula} [NOT] BETWEEN valor1 AND valor2}
					\item Teste de pertinência elemento-conjunto: \textbf{\textit{\textit{Cláusula} [NOT] IN}}
				\end{itemize}
			\end{frame}
		
			\begin{frame}{Incrementando a cláusula WHERE}
			
				\textbf{LIKE} pode ser traduzido como uma busca em um atributo que ``contém'', ``começa'' ou ``termina'' com um padrão, onde:
			
				\begin{itemize}
					\item \% = qualquer sequência de caracteres
					\item \_ = um único carácter
				\end{itemize}
				
				Entretanto, não é possível testar padrões em atributos DATETIME.
				
				\begin{block}{Exemplo}
					{\ttfamily
						SELECT pnome\\
						FROM Produto\\
						WHERE pnome LIKE ``p\%'' OR pnome LIKE ``\_a\%'' OR pnome LIKE ``\%p\%'' OR pnome LIKE ``\%a''
					}
				\end{block}
			\end{frame}
			
			\begin{frame}{Incrementando a cláusula WHERE}
			\BlankLine\BlankLine
				\begin{block}{Exemplo valores não nulos}
					{\ttfamily
						SELECT pnome\\
						FROM Produto\\
						WHERE pnome IS NOT NULL
					}
				\end{block}
	
				\begin{block}{Exemplo intervalo de valores}
					{\ttfamily
						SELECT pnome\\
						FROM Produto\\
						WHERE quantidade BETWEEN 10 and 100
					}
				\end{block}
			
				\begin{block}{Exemplo pertinência elemento-conjunto}
					{\ttfamily
						SELECT pnome\\
						FROM Produto\\
						WHERE id IN (1,3)
					}
				\end{block}
			\end{frame}
		\section{O que teremos na próxima aula?}
			\begin{frame}{\textit{The complex queries are coming...}}
				\begin{itemize}
					\item Consultas com mais de uma tabela: Junções em SQL
					\begin{itemize}
						\item União, Interseção, Diferença
						\item Subconsultas
						\item INTERSECT, EXCEPT e EXISTS
					\end{itemize}
					\item Agrupamento
						\begin{itemize}
							\item GROUP BY
							\item HAVING
						\end{itemize}
					\item Aula prática no laboratório
					\item Trabalho com entrega de relatório
				\end{itemize}
			\end{frame}
		\section{Exercícios}
			\begin{frame}{Exercícios para entrega}
				Entregar até a próxima aula a Lista de exercícios 4.
				
%				\begin{enumerate}
%					\item Encontre os produtos japoneses.
%					\item Encontre apenas as camisetas e ordene-ás por origem.
%					\item Encontre a camiseta de maior valor.
%					\item Encontre os produtos de id 1, 3 e 5.
%					\item Encontre o valor total gasto em produtos.
%					\item Encontre os produtos de origem japonesa com preço menor que $ 50.00 $.
%				\end{enumerate}
%				\BlankLine\BlankLine
%				Para responder as questões acima \textbf{utilize} a tabela do próximo slide.
			\end{frame}
%			\begin{frame}{Exercícios para entrega}
%				\begin{block}{\textbf{Produto}}
%					\begin{table}
%						\begin{tabular}{|c|c|l|l|c|}
%							\toprule
%							\textbf{idproduto} & \textbf{idvendedor} & \textbf{produto} & \textbf{origem} & \textbf{preco} \\ \midrule
%							        1          & 5                   &     Camiseta     &      Japão      & 39.99          \\ \midrule
%							        2          & 10                  &     Teclado      &      China      & 29.99          \\ \midrule
%							        3          & 101                 &     Monitor      &     Brasil      & 109.99         \\ \midrule
%							        4          & 5                   &     Bermuda      &      Japão      & 59.99          \\ \midrule
%							        5          & 2                   &     Camiseta     &    Portugal     & 29.99          \\ \bottomrule
%						\end{tabular}
%					\end{table}
%				\end{block}
%		\end{frame}
		\section{Bibliografia}
			\begin{frame}
				\frametitle{Bibliografia}
				\textbf{Básica:}\\
				ELMASRI, Ramez, NAVATHE, Shamkant B. Sistemas de banco de dados. 6 ed., 2011. Cap 4.
				
				\textcolor{ExecusharesGrey}{\small\hspace{1em} Total de exemplares disponíveis: 9 (CEPLAN), 10 (CCT), 2 (CEAVI). }
				
				\BlankLine\BlankLine
				\textbf{Leitura:}\\
				Capítulo 3, “Simple Queries” do livro \textbf{SQL
				for Web Nerds}, de Philip Greenspun
				\url{http://philip.greenspun.com/sql/}
			\end{frame}
		
		\frame{\acknowledgmentpage}

\end{document}